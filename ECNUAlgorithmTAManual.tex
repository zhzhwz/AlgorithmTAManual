\documentclass{article}

\usepackage{ctex}
\usepackage{geometry}
\usepackage[hidelinks]{hyperref}
\usepackage{xcolor}
\hypersetup{
  colorlinks,
  citecolor=violet,
  linkcolor=gray,
  urlcolor=blue}
\geometry{a4paper,scale=0.8}

\title{华东师范大学算法分析助教简易手册}
\author{\href{mailto:zhzhwz@foxmail.com}{zhzhwz@foxmail.com}}

\begin{document}

\maketitle
\tableofcontents
\newpage

\section{EOJ}

ECNU Online Judge (EOJ) 是由华东师范大学的老学长开发的在线评测系统。

\subsection{基础}

\subsubsection{EOJ 的网址}

\href{https://acm.ecnu.edu.cn/}{https://acm.ecnu.edu.cn/}

\subsubsection{EOJ 的基础用法}

请自行摸索或联系作者获取更多信息(我不觉得作者能给你比自行摸索能得到的更多的信息)。

\subsection{EOJ Polygon}

EOJ Polygon 是 EOJ 的『题目与比赛管理子系统』。

\subsubsection{如何获取 EOJ Polygon 权限}

\label{sssec_eoj_admin}

普通用户没有使用 EOJ Polygon 的权限。你需要联系 EOJ 的管理员以获取权限。以下是可能的寻找 EOJ 管理员的方式:

\begin{itemize}
    \item 在 EOJ 主页上寻找管理团队的联系方式(目前为 \href{mailto:acmsupport@admin.ecnu.edu.cn}{acmsupport@admin.ecnu.edu.cn}),联系时表明你的身份和需求(即,需要 EOJ Polygon 的权限),提供你的 EOJ 用户名。(推荐)
    \item 找你的课程老师,让他帮你联系算法竞赛团队的老师要权限
    \item 在 EOJ QQ 群(当前群号:691713742)中问谁是管理员(如果你是社牛可以考虑这个方案)
    \item 联系作者,作者帮你问(不推荐(如果实在不行就来问吧(叹气)))
\end{itemize}

\subsubsection{如何使用 EOJ Polygon}

在获取 Polygon 的权限后,你就可以在用户名的下拉菜单或者 EOJ 主页的最下方找到 Polygon 的入口了。EOJ 的主要创建人张羽戈在知乎写了一篇(并没有那么)详尽的文档,包含了几乎所有的 Polygon 的常见用法。详见:\href{https://zhuanlan.zhihu.com/p/59869879}{EOJ Polygon 使用指北 - 知乎}

\subsubsection{如何获取往届算法分析作业题和小测题的使用权限}

以下是可能的方法:

\begin{itemize}
    \item 找你的课程老师,让他给你权限。前提是你的老师拥有对应作业集或比赛的权限。具体的操作步骤为:进入Polygon -- 比赛管理 -- 找到对应的作业集或比赛 -- 进入 -- 修改管理权限 -- 输入并选择要授予权限的用户 -- 点击``必须在这里保存''。
    \item 找作者要权限。前提是作者有权限。
    \item 如果想要其他题目的权限,联系对应题目或作业集或比赛的管理员,或直接联系 EOJ 的管理员(在第 \ref{sssec_eoj_admin} 小小节介绍了如何联系)。
\end{itemize}

在获取作业集或比赛的权限后,可以通过 Request Clone 来获取其中题目的权限。详见\href{https://zhuanlan.zhihu.com/p/59869879}{EOJ Polygon 使用指北 - 知乎}。

\textbf{\textcolor{red}{注意:请务必在你创建的作业集或比赛中,在管理权限中加上你的老师。否则以后的助教就只能来找你要权限了。(你也可以在管理权限中加上作者(EOJ 用户名:zhzhwz))}}

\subsubsection{如何创建新的题目}

技术性的细节已经在 \href{https://zhuanlan.zhihu.com/p/59869879}{EOJ Polygon 使用指北 - 知乎} 中解释清楚了。如果你已经有一道题的题面和数据,你可以参照该文章的内容在 EOJ 上创建该题。如果你只有一个 idea,需要自己造数据、测试等,可以参见该文章:\href{https://oi-wiki.org/tools/testlib/}{testlib - OI Wiki}。(该文讲得并不是很清楚,你也可以用自己的方式造数据,或者自行查询网上关于 testlib 的教程。)

\section{大夏学堂}

大夏学堂有什么能讲的呢?自己摸索或者看大夏学堂的文档去吧。(主要是作者现在也登不进大夏学堂了)(欢迎补充该部分文档)

\section{后记}

若还有本文未能解决的问题,请先尝试自行解决,鼓励在解决之后在本文的\href{https://github.com/zhzhwz/AlgorithmTAManual}{仓库}提交 issue 或 pr 将对应内容加入手册中。若未能解决,可同样在本文的仓库提交 issue 或直接联系作者\href{mailto:zhzhwz@foxmail.com}{zhzhwz@foxmail.com}。 

\end{document}
